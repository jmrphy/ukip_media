\documentclass[12pt,article]{article}
\usepackage{lmodern}
\usepackage{amssymb,amsmath}
\usepackage{ifxetex,ifluatex}
\usepackage{fixltx2e} % provides \textsubscript
\ifnum 0\ifxetex 1\fi\ifluatex 1\fi=0 % if pdftex
  \usepackage[T1]{fontenc}
  \usepackage[utf8]{inputenc}
\else % if luatex or xelatex
  \ifxetex
    \usepackage{mathspec}
    \usepackage{xltxtra,xunicode}
  \else
    \usepackage{fontspec}
  \fi
  \defaultfontfeatures{Mapping=tex-text,Scale=MatchLowercase}
  \newcommand{\euro}{€}
\fi
% use upquote if available, for straight quotes in verbatim environments
\IfFileExists{upquote.sty}{\usepackage{upquote}}{}
% use microtype if available
\IfFileExists{microtype.sty}{%
\usepackage{microtype}
\UseMicrotypeSet[protrusion]{basicmath} % disable protrusion for tt fonts
}{}
\usepackage[margin=1.25in]{geometry}
\usepackage{graphicx}
\makeatletter
\def\maxwidth{\ifdim\Gin@nat@width>\linewidth\linewidth\else\Gin@nat@width\fi}
\def\maxheight{\ifdim\Gin@nat@height>\textheight\textheight\else\Gin@nat@height\fi}
\makeatother
% Scale images if necessary, so that they will not overflow the page
% margins by default, and it is still possible to overwrite the defaults
% using explicit options in \includegraphics[width, height, ...]{}
\setkeys{Gin}{width=\maxwidth,height=\maxheight,keepaspectratio}
\ifxetex
  \usepackage[setpagesize=false, % page size defined by xetex
              unicode=false, % unicode breaks when used with xetex
              xetex]{hyperref}
\else
  \usepackage[unicode=true]{hyperref}
\fi
\hypersetup{breaklinks=true,
            bookmarks=true,
            pdfauthor={},
            pdftitle={Does Public Support for UKIP Drive Media Coverage or Does Media Coverage Drive Support for UKIP?},
            colorlinks=true,
            citecolor=blue,
            urlcolor=blue,
            linkcolor=blue,
            pdfborder={0 0 0}}
\urlstyle{same}  % don't use monospace font for urls
\setlength{\parindent}{0pt}
\setlength{\parskip}{6pt plus 2pt minus 1pt}
\setlength{\emergencystretch}{3em}  % prevent overfull lines
\setcounter{secnumdepth}{0}

%%% Change title format to be more compact
\usepackage{titling}
\setlength{\droptitle}{-2em}
  \title{Does Public Support for UKIP Drive Media Coverage or Does Media Coverage
Drive Support for UKIP?}
  \pretitle{\vspace{\droptitle}\centering\huge}
  \posttitle{\par}
  \author{Justin Murphy\\University of Southampton}
  \preauthor{\centering\large\emph}
  \postauthor{\par}
  \date{}
  \predate{}\postdate{}


\usepackage{dcolumn}
\usepackage{setspace}


\begin{document}

\maketitle


\begin{abstract}
Previous research suggests media attention may cause support for populist right-wing parties, but this finding is debated and extant evidence remains limited to proportional representation systems in which such an effect would be most likely. At the same time, in the United Kingdom's first-past-the-post system, an ongoing political and regulatory debate revolves around whether the media give disproportionate coverage to the populist right-wing UK Independence Party (UKIP). Thus, we use a mixed-methods approach to investigate the causal dynamics of UKIP support and media coverage as an especially valuable case. Vector autoregression (VAR) using monthly, aggregate time-series data from 2004 to September 2015 provides additional evidence, from a new and less-likely institutional environment, consistent with the model that media drive party support, but not vice-versa. Additionally, qualitative investigation of
the dynamics suggests that in at least two key periods of stagnating or declining support for UKIP, media coverage increased and was followed by increases in public support. Overall the findings show that media coverage can and does
appear to drive public support in a substantively important fashion irreducible
to previous levels of public support.
\end{abstract}
\doublespacing

\section{Introduction}\label{introduction}

If the visibility of a political party in the media shapes the public
support it receives, then the degree to which the media gives attention
to different political parties can have significant implications for
democracy. In the United Kingdom, critics allege that the media pays
disproportionate attention to the populist, right-wing UK Independence
Party (UKIP) but media elites claim that media coverage of UKIP is
driven by increasing public support for the party. Descriptively, media
attention to UKIP is greater than that given to other, similarly small
parties on the right as well as the left (Goodwin and Ford, 2013;
Stevenson, 2014; Soussi, 2014), but UK media regulator Ofcom as well as
the BBC have publicly defended the attention paid to UKIP on grounds of
public support for the party (Sweeney, 2015; Wintour, 2015). Implied in
this elite reasoning is a causal model, namely that public support
drives media coverage rather than vice-versa.

Yet previous research from multiparty, proportional representation
systems suggests that media coverage drives public support

We present the first statistical time-series analysis testing the degree
to which media coverage of UKIP is driven by public support for UKIP
and/or vice versa. In particular, I gather monthly time-series of public
support for UKIP from Ipsos MORI's voting intention polls and a monthly
count of UKIP mentions in all UK National Newspapers (drawn from the
databse Nexis). I begin with a series of econometric analyses to
investigate whether media coverage drives support, support drives media
coverage, or both. First, vector-autoregression (VAR) is used as a
straightforward and relatively atheoretical way to document the stylized
facts of the causal dynamics. Then, separate error-correction models are
estimated as an alternative approach to the question making different
assumptions. Finally, I provide a brief qualitative examination of the
time-series. Both econometric techniques and the qualitative evidence
converge on the conclusion that the relationship between public support
and media coverage of UKIP is one of positive feedback: while public
support is positively correlated with future levels of media coverage,
media coverage is also independently correlated with future increases in
public support. Qualitative exploration of these dynamics identify at
least two key time periods in which support for UKIP was decreasing but
media coverage increased and may have caused support for UKIP to
increase: a surge in media coverage in the second half of 2012, not
triggered by any public support, was followed by a surge of support
which came in the first half of 2013. Then, despite \emph{decreasing}
support throughout 2013, media coverage increases briefly before surging
in the second half of 2013 until the middle of 2014, apparently helping
to maintain current levels of support.

\pagebreak

\includegraphics{ukip_media_files/figure-latex/unnamed-chunk-1-1.pdf}
\includegraphics{ukip_media_files/figure-latex/unnamed-chunk-1-2.pdf}

\begin{figure}[htbp]
\centering
\includegraphics{ukip_media_files/figure-latex/unnamed-chunk-2-1.pdf}
\caption{Dynamics of UKIP Support and Media Coverage}
\end{figure}

\section{Background and Literature
Review}\label{background-and-literature-review}

The crucial and controversial question--and the motivation for this
article--is whether the quantity of UKIP's media coverage represents a
form of media bias with negative implications for democracy, or if the
media's fascination with UKIP is merely a reasonable or even healthy
response to a newly rising political party. If disproportionate media
coverage is not simply responsive to public opinion but effectively
driving public opinion and shaping outcomes such as elections, as some
argue (Soussi, 2014), then clearly the media's disproportionate coverage
of UKIP is at best normatively problematic and at worst complicit in the
legitimation and empowerment of what is alleged to be UKIP's coded
racism and proto-fascism (Webb, 2014; Syal, 2015).

While a great deal of research documents various aspects of the
relationship between media and right-wing populist parties, little is
known specifically about the effects of the quantity of party-specific
media coverage on aggregate party support from a dynamic perspective.
The closest previous research comes to this particular question is work
by (Boomgaarden and Vliegenthart, 2007, 2009), who find that greater
media coverage of immigration issues is positively associated with
support for anti-immigration parties. Other research has studied the
dynamics of individual-level exposure to media coverage and perceptions
of right-wing politicians (Bos et al., 2011). Still, currently extant
research tells us surprisingly little about the causal dynamics of
public support and the quanitity of media coverage of right-wing
populist parties. In part, this may be because the application of
time-series techniques to aggregate media data remains relatively
under-explored (Vliegenthart, 2014a).

H1: Coverage drives support Most of the literature concerning the
visibility-support nexus investigates the impact of media coverage on
public opinion and electoral support. To our knowledge, this has only
been studied in multi-party proportional systems such as Belgium
(WALGRAVE and SWERT, 2004), the Netherlands and Germany (Boomgaarden and
Vliegenthart, 2007; Vliegenthart and Boomgaarden, 2010; Vliegenthart et
al., 2012). In all of these cases, however, the authors found a positive
effect of media coverage on the intention to vote for right-wing and
anti-immigration parties although some have called these findings into
question on the basis of methodological concerns (Pauwels, 2010).

Other scholars have addressed the question indirectly. In his study on
the diffusion of the populist message in the media, (Rooduijn, 2014)
hypothesises that the electoral success of populist parties affects the
degree and acceptability of populism in the media. Other authors in the
literature explore the theoretical connection of media coverage and the
rise of populist and right-wing parties, but offer little or no
empirical evidence for the claim (Art, 2007; Mudde, 2013). Nevertheless,
there is a considerable body of literature which posits particular
mechanisms which may offer a causal explanation for how the quantity of
media coverage can increase support for a political party. The primary
explanation revolves around issue saliency and the aligning of a party
or a party leader with those salient issues (Brug et al., 2006; Cushion
et al., 2015; Dennison and Goodwin, 2015). In the case of UKIP, the
party was strongly aligned with the issue of immigration and the
European Union, and the increasing prominence of these issues in the
media drove both coverage and support for the party. (WALGRAVE and
SWERT, 2004) offer a similar analysis of the Vlaams Blok in Belgium,
concluding that the media are at least part responsible for the growth
of the right-wing party. Although that study does conclude in support of
this relationship, they do not commit themselves to arguing a causal
relationship.

H1: Increases in media coverage lead to increased public support,
controlling for previous levels of public support.

H2: Support drives coverage Alternatively, the causal arrow could run
the other way, where the support that a party receives increases the
amount of media coverage. As (Vliegenthart and Boomgaarden, 2010)
consider, this could be related to the power and position of political
figures, referencing studies from both America (Sellers and Schaffner,
2007) and Switzerland (Tresch, 2009) which highlight how the standing of
a political actor influences media attention. Another possibility is
that media coverage depends on the dynamics of the party itself
(Pauwels, 2010): in our case, the relative acceptability of UKIP's
agenda as opposed to other British populist parties such as the British
National Party may have contributed to the rise in coverage, as well as
the popularity and charisma of Nigel Farage. Finally, political polling
and the reporting of political polls is ubiquitous in British media,
including tabloid papers running polls of their own readers. It could be
the case that increasing media coverage is simply reflective of their
standing in polls, or that there is a positive feedback mechanism
operating between media coverage and polling. We therefore test this
possibility, hypothesising that

H2: Increases in public support for UKIP lead to increased media
coverage, controlling for previous levels of media coverage

\section{Data, Method, and Research
Strategy}\label{data-method-and-research-strategy}

To measure public support for UKIP, I gathered monthly aggregate polling
data on vote intentions from Ipsos MORI (Ipsos-MORI, n.d.).
Specifically, I constructed a variable from the percentage reporting an
intention to vote for UKIP according to the Ipsos MORI poll closest to
the middle of each month. For most months, this was straightforward
because the Ipsos MORI poll is approximately monthly. For months with
multiple polls, I used the poll closest to the middle of the month. For
the very few months with no poll or a poll at the border between the
previous or following month, the value was counted as missing and then
all missing values were linearly interpolated. To measure media coverage
of UKIP, I gathered monthly counts of all UK national newspaper reports
mentioning either ``UKIP'' or ``UK Independence Party'' from the
database Nexis (Anon.,
n.d.).\footnote{Duplicate articles defined by Nexis's definition of high similarity were excluded.}

Econometric techniques are used to test for, and distinguish the
ordering of, potential causal dynamics between media coverage and public
support for UKIP. A brief qualitative historical analysis of these
dynamics will be used to better understand a potential causal process.
In particular, the substantive nature of the puzzle at hand requires the
identification of a causal narrative. Even with econometric evidence
suggesting an independent causal effect from either one to the other, it
would not be clear whether the historical unfolding of these causal
dynamics implies a problem for democracy. We are not only interested in
whether media coverage amplifies exogenous increases in support--this
would be an important but not necessarily problematic finding from a
democratic perspective--but whether increases in media coverage have
generated support for UKIP despite low, stagnant, or decreasing levels
of support.

\section{Analysis}\label{analysis}

\subsection{VAR}\label{var}

Because both variables are non-stationary, vector autoregression is
estimated with first differences of each variable. Optimal lag length is
determined by the Aikeke Information Criterion to be to be VAR(3). The
model includes a constant and a trend term. Diagnostics suggest that
using the log of each variable before differencing reduces
heteroskedasticity and serial correlation of errors. Because VAR models
have many paramaters to begin with, monthly indicators controlling for
seasonality absorb crucial degrees of freedom and so are excluded in the
intitial models but added in subsequent models. The models displayed
here all pass the standard ARCH-LM and Portmanteau tests for
non-constant error variance and serial correlation of errors,
respectively. Finally, diagnostics show no evidence of significant
temporal instability (see Supplementary Information).

Surprisingly, initial VAR results show little evidence that changes in
public support predict media coverage, but significant evidence that
media coverage drives public support. As the numerical results and the
Impulse Response plots show, there is no statistically discernable
correlation between past changes in public support and changes in media
coverage, whereas past changes in media coverage have a statistically
significant correlation with changes in public support. Granger
causality tests support this interpretation, with only the latter
relationship nearing conventional cutoffs of statistical significance
(p\textless{}.08).

After including monthly indicators, however, the results reverse: while
the coefficients reflecting the correlation between past changes in
media coverage and public support do not change noticeably, they lose
statistical significance, whereas the coefficients for the other model
become significant and pass the test for Granger causality. Because the
coefficients reflecting the correlation between past changes in media
coverage and public support remain signed as predicted, the increased
standard errors do not necessarily reflect the absence of a relationship
but possibly only a lack of degrees of freedom due to the introduction
of the seasonality indicators.

Additionally, there are limitations of the data which may make it
difficult to identify causal effects in a VAR approach. First, it is
possible that monthly measures are too infrequent to capture causal
effects if the real lag between effects is more shorter than one month.
Importantly, structural tests on all models suggest strong evidence of
instantaneous causality.

Taken together, VAR results suggest qualified evidence for both
directions of causality. While the results are sensitive to the
specification, the results are consistent with the possibility that both
variables drive each other, but that highly robust evidence of this in
one model is not possible due to the nature of the data and the
high-paramater demands of the VAR approach. \pagebreak

\% Table created by stargazer v.5.1 by Marek Hlavac, Harvard University.
E-mail: hlavac at fas.harvard.edu \% Date and time: Mon, Oct 12, 2015 -
17:40:12

\begin{table}[!htbp] \centering 
  \caption{} 
  \label{} 
\begin{tabular}{@{\extracolsep{5pt}}lcc} 
\\[-1.8ex]\hline 
\hline \\[-1.8ex] 
 & \multicolumn{2}{c}{\textit{Dependent variable:}} \\ 
\cline{2-3} 
\\[-1.8ex] & \multicolumn{2}{c}{y} \\ 
\\[-1.8ex] & (1) & (2)\\ 
\hline \\[-1.8ex] 
 UKIP.Articles.l1 & 0.168$^{*}$ & $-$0.387$^{***}$ \\ 
  & (0.095) & (0.094) \\ 
  & & \\ 
 UKIP.Vote.l1 & $-$0.483$^{***}$ & $-$0.007 \\ 
  & (0.106) & (0.105) \\ 
  & & \\ 
 UKIP.Articles.l2 & 0.160$^{*}$ & $-$0.341$^{***}$ \\ 
  & (0.093) & (0.092) \\ 
  & & \\ 
 UKIP.Vote.l2 & $-$0.261$^{**}$ & $-$0.085 \\ 
  & (0.103) & (0.102) \\ 
  & & \\ 
 UKIP.Articles.l3 & 0.161$^{*}$ & $-$0.196$^{**}$ \\ 
  & (0.091) & (0.090) \\ 
  & & \\ 
 UKIP.Vote.l3 & $-$0.111 & $-$0.088 \\ 
  & (0.096) & (0.096) \\ 
  & & \\ 
 const & 0.012 & 0.027 \\ 
  & (0.080) & (0.079) \\ 
  & & \\ 
 trend & 0.00000 & $-$0.0002 \\ 
  & (0.001) & (0.001) \\ 
  & & \\ 
 General.Elections & $-$0.099 & 0.185 \\ 
  & (0.273) & (0.271) \\ 
  & & \\ 
 EU.Elections & 0.465 & 0.894$^{***}$ \\ 
  & (0.296) & (0.294) \\ 
  & & \\ 
\hline \\[-1.8ex] 
Observations & 137 & 137 \\ 
R$^{2}$ & 0.148 & 0.233 \\ 
Adjusted R$^{2}$ & 0.087 & 0.179 \\ 
Residual Std. Error (df = 127) & 0.445 & 0.443 \\ 
F Statistic (df = 9; 127) & 2.448$^{**}$ & 4.284$^{***}$ \\ 
\hline 
\hline \\[-1.8ex] 
\textit{Note:}  & \multicolumn{2}{r}{$^{*}$p$<$0.1; $^{**}$p$<$0.05; $^{***}$p$<$0.01} \\ 
\end{tabular} 
\end{table}

\% Table created by stargazer v.5.1 by Marek Hlavac, Harvard University.
E-mail: hlavac at fas.harvard.edu \% Date and time: Mon, Oct 12, 2015 -
17:40:13

\begin{table}[!htbp] \centering 
  \caption{Granger Causality Tests} 
  \label{} 
\begin{tabular}{@{\extracolsep{5pt}} ccc} 
\\[-1.8ex]\hline \\[-1.8ex] 
 & Support & Articles \\ 
\hline \\[-1.8ex] 
P-value & $0.750$ & $0.060$ \\ 
DF1 & $3$ & $3$ \\ 
DF2 & $254$ & $254$ \\ 
F-test & $0.405$ & $2.499$ \\ 
\hline \\[-1.8ex] 
\end{tabular} 
\end{table}

\includegraphics{ukip_media_files/figure-latex/unnamed-chunk-6-1.pdf}
\includegraphics{ukip_media_files/figure-latex/unnamed-chunk-6-2.pdf}

\section{Process-tracing}\label{process-tracing}

\section{This paper has investigated a simple claim by the UK's
regulatory authorities and national broadcaster: that the increased
media coverage of UKIP was justified by the party's increased poll
ratings. We have found that, on the contrary, on average media coverage
preceded an increase in poll ratings controlling for previous media
coverage, rather than media coverage increasing controlling for previous
public support. {[}some stuff on
findings{]}.}\label{this-paper-has-investigated-a-simple-claim-by-the-uks-regulatory-authorities-and-national-broadcaster-that-the-increased-media-coverage-of-ukip-was-justified-by-the-partys-increased-poll-ratings.-we-have-found-that-on-the-contrary-on-average-media-coverage-preceded-an-increase-in-poll-ratings-controlling-for-previous-media-coverage-rather-than-media-coverage-increasing-controlling-for-previous-public-support.-some-stuff-on-findings.}

We have made three contributions with this study. Firstly, this is the
first paper, in our knowledge, to address the visibility-support nexus
in the context of the United Kingdom and a majoritarian system; previous
research has primarily focused on other Western European democracies
such as Belgium, the Netherlands and Germany. Despite the change in
political system, our findings support those of (Vliegenthart and
Boomgaarden, 2010; Vliegenthart et al., 2012), and find that the media
can and have independently generated support for UKIP. We have left
aside the question of leader effects, given previously ambiguous
findings. There is also reason to believe that media dynamics are
different in proportional systems, being more diverse in their coverage
than in majoritarian systems ({\textbf{???}}).

Secondly, we have also contributed methodologically in two ways. We have
offered qualitative evidence for our findings that, at least in this
case, the media have generated support for a radical right-wing party.
Previous research has offered only statistical evidence, which may not
pick up questions relating to the historical narrative of the party in
question. We address this gap here and find that the results are still
robust. We have also contributed to a growing body of literature that
uses time-series methods to address questions relating to the media
(Vliegenthart, 2014b).

Perhaps most importantly, these findings are of significance to
contemporary public debate in the UK concerning the role of the media
and the perceived unfair coverage of UKIP. Some have argued that the
media coverage of UKIP is justified due to its public support. The
findings here, on the other hand, suggest that the causal arrow points
the other way: that the media coverage drove the support of UKIP
independent of its previous poll ratings. As with all studies, there are
limitations and areas for future research. Firstly, we do not undertake
any form of content analysis to get at the actual content of the
coverage in question, but only look at the quantity of articles. It is
possible that, by disaggregating the coverage further, different types
of coverage change the findings; it would also be interesting to see
whether how positive or negative the coverage is matters for changing
public opinion. Similarly, we do not disaggregate between types of
paper, such as broadsheet and tabloid, which offer different coverage
and target a different readership.

We also only focus on print media. This means that we have not accounted
for the effect of visual and social media which may be contributing to
this relationship. Despite these limitations, this paper provides a
contribution to the continuing and growing debate concerning the media's
role in the growth of political parties and the wider ramifications for
democratic debate.

\pagebreak

\begin{figure}[htbp]
\centering
\includegraphics{ukip_media_files/figure-latex/supplementary-1.pdf}
\caption{}
\end{figure}

\pagebreak

\section{References}\label{references}

\raggedright

Anon. (n.d.) ``Nexis.''

Art, David. (2007) ``Reacting to the Radical Right Lessons from Germany
and Austria.'' \emph{Party Politics} 13:331--349.
\url{http://ppq.sagepub.com/content/13/3/331} (Accessed October 10,
2015).

Boomgaarden, Hajo G, and Rens Vliegenthart. (2007) ``Explaining the rise
of anti-immigrant parties: The role of news media content.''
\emph{Electoral Studies} 26:404--417.

Boomgaarden, Hajo G, and Rens Vliegenthart. (2009) ``How news content
influences anti-immigration attitudes: Germany, 19932005.''
\emph{European Journal of Political Research} 48:516--542.

Bos, Linda, Wouter van der Brug, and Claes de Vreese. (2011) ``How the
Media Shape Perceptions of Right-Wing Populist Leaders.''
\emph{Political Communication} 28:182--206.

Brug, Wouter van der, Holli A. Semetko, and Patti M. Valkenburg. (2006)
``Media Priming in a Multi-Party Context: A Controlled Naturalistic
Study in Political Communication.'' \emph{Political Behavior}
29:115--141.
\url{http://link.springer.com/article/10.1007/s11109-006-9020-7}
(Accessed October 3, 2015).

Cushion, Stephen, Richard Thomas, and Oliver Ellis. (2015)
``Interpreting UKIP's `Earthquake' in British Politics: UK Television
News Coverage of the 2009 and 2014 EU Election Campaigns.'' \emph{The
Political Quarterly} 86:314--322.
\url{http://onlinelibrary.wiley.com/doi/10.1111/1467-923X.12169/abstract}
(Accessed October 1, 2015).

Dennison, James, and Matthew Goodwin. (2015) ``Immigration, Issue
Ownership and the Rise of UKIP.'' \emph{Parliamentary Affairs}
68:168--187. \url{http://pa.oxfordjournals.org/content/68/suppl_1/168}
(Accessed October 1, 2015).

Goodwin, Matthew, and Robert Ford. (2013) ``Just how much media coverage
does UKIP get?'' \emph{New Statesman}.
\url{http://www.newstatesman.com/politics/2013/11/just-how-much-media-coverage-does-ukip-get}.

Ipsos-MORI. (n.d.) ``Voting Intention in Great Britain: Recent Trends.''
\url{https://www.ipsos-mori.com/researchpublications/researcharchive/poll.aspx?oItemId=107\&view=wide}.

Mudde, Cas. (2013) ``Three decades of populist radical right parties in
Western Europe: So what?'' \emph{European Journal of Political Research}
52:1--19.

Pauwels, Teun. (2010) ``Reassessing conceptualization, data and
causality: A critique of Boomgaarden and Vliegenthart's study on the
relationship between media and the rise of anti-immigrant parties.''
\emph{Electoral Studies} 29:269--275.
\url{http://www.sciencedirect.com/science/article/pii/S0261379410000120}
(Accessed October 10, 2015).

Rooduijn, Matthijs. (2014) ``The Mesmerising Message: The Diffusion of
Populism in Public Debates in Western European Media.'' \emph{Political
Studies} 62:726--744.
\url{http://onlinelibrary.wiley.com/doi/10.1111/1467-9248.12074/abstract}
(Accessed October 2, 2015).

Sellers, Patrick J., and Brian F. Schaffner. (2007) ``Winning Coverage
in the U.S. Senate.'' \emph{Political Communication} 24:377--391.
\url{http://dx.doi.org/10.1080/10584600701641516} (Accessed October 11,
2015).

Soussi, Alasdair. (2014) ``Did British media help the UKIP win EU
poll?''
\url{http://www.aljazeera.com/indepth/features/2014/06/did-british-media-help-ukip-win-eu-poll-20146313346918679.html}.

Stevenson, Alex. (2014) ``Caroline Lucas points finger at media's Farage
obsession.''
\url{http://www.politics.co.uk/news/2014/05/29/ukip-victory-caroline-lucas-points-finger-at-media-s-farage}.

Sweeney, Mark. (2015) ``BBC prepares to boost Ukip coverage as it ranks
it a larger party in election.''
\url{http://www.theguardian.com/media/2015/jan/15/bbc-prepares-to-boost-ukip-coverage-as-it-ranks-it-a-larger-party-in-election}.

Syal, Rajeev. (2015) ``Ukip faces crisis after suspensions and racism
claims.''
\url{http://www.theguardian.com/politics/2015/mar/20/ukip-faces-crisis-two-parliamentary-candidates-suspended-one-resigns}.

Tresch, Anke. (2009) ``Politicians in the Media: Determinants of
Legislators' Presence and Prominence in Swiss Newspapers.'' \emph{The
International Journal of Press/Politics} 14:67--90.
\url{http://hij.sagepub.com/content/14/1/67} (Accessed October 11,
2015).

Vliegenthart, Rens. (2014a) ``Moving up. Applying aggregate level time
series analysis in the study of media coverage.'' \emph{Quality \&
Quantity} 48:2427--2445.

Vliegenthart, Rens. (2014b) ``Moving up. Applying aggregate level time
series analysis in the study of media coverage.'' \emph{Quality \&
Quantity} 48:2427--2445.

Vliegenthart, Rens, and Hajo G. Boomgaarden. (2010) ``Why the media
matter after all: A response to Pauwels.'' \emph{Electoral Studies}
29:719--723.
\url{http://www.sciencedirect.com/science/article/pii/S0261379410000855}
(Accessed October 11, 2015).

Vliegenthart, Rens, Hajo G. Boomgaarden, and Joost Van Spanje. (2012)
``Anti-Immigrant Party Support and Media Visibility: A Cross-Party,
Over-Time Perspective.'' \emph{Journal of Elections, Public Opinion and
Parties} 22:315--358.
\url{http://dx.doi.org/10.1080/17457289.2012.693933} (Accessed October
10, 2015).

WALGRAVE, STEFAAN, and KNUT DE SWERT. (2004) ``The Making of the (Issues
of the) Vlaams Blok.'' \emph{Political Communication} 21:479--500.
\url{http://dx.doi.org/10.1080/10584600490522743} (Accessed October 10,
2015).

Webb, Robert. (2014) ``Ukip trades in the language of fear and division.
The left must not humour its anti-politics crusade.''
\url{http://www.newstatesman.com/politics/2014/05/ukip-trades-language-fear-and-division-left-must-not-humour-its-anti-politics}.

Wintour, Patrick. (2015) ``Ofcom deals blow to Greens election debate
hopes but boosts Ukips.''
\url{http://www.theguardian.com/politics/2015/jan/08/ofcom-blow-green-party-election-debate-boost-ukips}.

\end{document}
