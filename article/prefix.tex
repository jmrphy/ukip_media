\begin{abstract}
Previous research suggests media attention may cause support for populist right-wing parties, but this finding is debated and extant evidence remains limited to proportional representation systems in which such an effect would be most likely. At the same time, in the United Kingdom's first-past-the-post system, an ongoing political and regulatory debate revolves around whether the media give disproportionate coverage to the populist right-wing UK Independence Party (UKIP). Thus, we use a mixed-methods approach to investigate the causal dynamics of UKIP support and media coverage as an especially valuable case. Vector autoregression (VAR) using monthly, aggregate time-series data from 2004 to September 2015 provides new evidence consistent with a model in which media coverage drives party support, but party support does not drive media coverage. Additionally, qualitative investigation of the dynamics suggests that in at least two key periods of stagnating or declining support for UKIP, media coverage increased and was followed by increases in public support. Overall the findings show that media coverage can and does appear to drive public support in a substantively important fashion irreducible to previous levels of public support, even in a national institutional environment least supportive of such an effect. The findings have direct and troubling implications for contemporary political and regulatory debates in the United Kingdom and potentially liberal democracies more generally.
\end{abstract}
\doublespacing